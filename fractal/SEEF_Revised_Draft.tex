\documentclass[11pt,letterpaper]{article}
\usepackage{amsmath}
\usepackage{amssymb}
\usepackage{physics}
\usepackage{hyperref}
\usepackage{graphicx}
\usepackage{color}
\usepackage{tikz}
\usepackage{booktabs}
\usepackage{multirow}
\usetikzlibrary{shapes,arrows.meta,positioning}

\title{SEEF: Steven Elliott's Fractal Universal Cosmological Kinematics \\
       \large Navier-Stokes Equations Rederived via Maxwell Quaternions}
\author{Steven E. Elliott}
\date{January 26, 2026}

\begin{document}

\maketitle

\begin{abstract}
I introduce \textbf{SEEF (Steven Elliott's Fractal Universal Cosmological Kinematics)}, a reformulation of physics that begins by rederiving the Navier-Stokes equations using Maxwell equation quaternion objects instead of classical point-particle billiards. This yields the \textbf{Steven Elliott Equations}—a unified fluid-electromagnetic framework that I hypothesize exhibits fractal self-similarity across all scales. If the fractal nature of Navier-Stokes dynamics replicates the astro-quantum patterns observed in nature, then gravity, dark matter, and cosmological phenomena emerge naturally from scale-invariant fluid dynamics rather than requiring separate fundamental forces. This paper presents the ontological derivation of SEEF and outlines computational validation via direct numerical simulation.
\end{abstract}

\tableofcontents
\newpage

\section{Introduction: Rethinking the Foundation}

\subsection{The Problem with Classical Derivations}

The Navier-Stokes equations are traditionally derived by taking the continuum limit of classical billiard-ball collisions—point particles with positions and momenta interacting via contact forces. This derivation obscures a crucial possibility: \textit{what if the fundamental objects aren't scalar billiard balls, but oriented electromagnetic entities?}

Maxwell's equations describe fields, but they can be reformulated in terms of quaternion objects that carry both charge density and orientation. If we rederive the Navier-Stokes equations starting from these Maxwell quaternion objects instead of classical particles, we obtain a new set of fluid equations—the \textbf{Steven Elliott Equations}—that naturally incorporate electromagnetic structure.

\subsection{The Fractal Hypothesis}

The central hypothesis of SEEF is:

\begin{center}
\textit{The Navier-Stokes equations, when derived from Maxwell quaternions, exhibit fractal self-similarity across all scales.}
\end{center}

If this is true, then:
\begin{itemize}
    \item Turbulent flows at human scales
    \item Stellar/galactic dynamics
    \item Quantum particle interactions
\end{itemize}
are all manifestations of the \textit{same underlying fluid dynamics}, differing only by scale factors and boundary conditions.

This fractal structure would explain the observed astro-quantum similarities without invoking separate fundamental forces. Gravity, in this view, is not a distinct interaction but rather an emergent property of fractal Navier-Stokes dynamics operating at larger scales.

\subsection{Validation Strategy: Simulation First}

Unlike traditional theoretical physics that relies primarily on observational validation, SEEF can be tested directly through numerical simulation:

\begin{enumerate}
    \item Implement the Steven Elliott Equations as a computational fluid dynamics (CFD) solver
    \item Run simulations at different scales with appropriately scaled parameters
    \item Check whether galactic rotation curves, dark matter distributions, and quantum scattering emerge from the same fractal NS dynamics
    \item Compare simulation outputs to both astrophysical observations and quantum measurements
\end{enumerate}

The rest of this paper develops the mathematical framework and simulation roadmap.

\newpage

\section{Part I: Deriving the Steven Elliott Equations}

\subsection{Starting Point: Maxwell Quaternions}

Instead of starting with point particles, we begin with the quaternion formulation of electromagnetism. Each "fluid element" is represented by a quaternion:

\begin{equation}
q = (s, \mathbf{v})
\end{equation}

where:
\begin{itemize}
    \item $s$ = scalar charge density $\rho$
    \item $\mathbf{v}$ = 3D vector orientation/dipole moment
\end{itemize}

The electromagnetic field is:
\begin{equation}
\mathbf{F} = \mathbf{E} + ic\mathbf{B}
\end{equation}

This is the fundamental object from which we will derive fluid dynamics.

\subsection{Quaternion Charge-Current Continuity}

The conservation of quaternion charge-current is:

\begin{equation}
\boxed{\frac{\partial \rho_q}{\partial t} + \nabla \cdot (\rho_q \mathbf{v}_q) = 0}
\end{equation}

where the quaternion charge-current density is:
\begin{equation}
\rho_q = \rho + i\frac{\mathbf{J}}{c}
\end{equation}

This replaces the classical mass continuity equation. Notice it includes both charge conservation and current conservation in a single equation.

\subsection{Quaternion Momentum Evolution}

The momentum density evolves according to:

\begin{equation}
\boxed{\frac{\partial \mathbf{g}_q}{\partial t} + \nabla \cdot (\mathbf{v}_q \mathbf{g}_q) = -\nabla p_q + \eta \nabla^2 \mathbf{v}_q + \mathbf{f}_q}
\end{equation}

This is the \textit{Navier-Stokes equation rederived from Maxwell quaternions}.

Key terms:
\begin{itemize}
    \item $\mathbf{g}_q$ = quaternion momentum density (includes both mechanical and electromagnetic momentum)
    \item $p_q$ = quaternion pressure (scalar + vector pressure components)
    \item $\eta$ = quaternion viscosity (electromagnetic dissipation mechanism)
    \item $\mathbf{f}_q$ = quaternion Lorentz force density
\end{itemize}

\subsection{Quaternion Field Dynamics}

The electromagnetic field evolves via:

\begin{equation}
\boxed{\nabla \mathbf{F} = -\frac{\partial \mathbf{F}}{\partial (c t)} + \frac{\rho_q}{\epsilon_0}}
\end{equation}

This replaces Maxwell's equations and couples directly to the fluid evolution.

\subsection{The Steven Elliott Equations: Summary}

The complete system is:

\begin{align}
\text{Continuity:} \quad & \frac{\partial \rho_q}{\partial t} + \nabla \cdot (\rho_q \mathbf{v}_q) = 0 \\[8pt]
\text{Momentum:} \quad & \frac{\partial \mathbf{g}_q}{\partial t} + \nabla \cdot (\mathbf{v}_q \mathbf{g}_q) = -\nabla p_q + \eta \nabla^2 \mathbf{v}_q + \mathbf{f}_q \\[8pt]
\text{Field:} \quad & \nabla \mathbf{F} = -\frac{\partial \mathbf{F}}{\partial (c t)} + \frac{\rho_q}{\epsilon_0}
\end{align}

where:
\begin{align}
\rho_q &= \rho + i\frac{\mathbf{J}}{c} \\
\mathbf{F} &= \mathbf{E} + ic\mathbf{B}
\end{align}

These are the \textbf{Steven Elliott Equations}—Navier-Stokes + Maxwell unified via quaternions.

\subsection{Relation to Classical NS and Maxwell}

\textbf{Classical Navier-Stokes limit:}

When electromagnetic fields are weak ($\mathbf{F} \to 0$), the quaternion components decouple and we recover:
\begin{align}
\frac{\partial \rho}{\partial t} + \nabla \cdot (\rho \mathbf{v}) &= 0 \\
\frac{\partial \mathbf{g}}{\partial t} + \nabla \cdot (\mathbf{v} \mathbf{g}) &= -\nabla p + \eta \nabla^2 \mathbf{v}
\end{align}

\textbf{Classical Maxwell limit:}

When fluid velocities are small ($\mathbf{v}_q \to 0$), the field equation decouples and we recover Maxwell's equations in quaternion form.

\textbf{New regime:}

When both electromagnetic and fluid effects are strong, the Steven Elliott Equations describe a coupled electro-hydrodynamic system that classical theory cannot capture.

\newpage

\section{Part II: The Fractal Hypothesis}

\subsection{Scale Invariance of NS Dynamics}

The Navier-Stokes equations are known to exhibit complex, scale-dependent behavior (turbulence, eddies at all scales, etc.). The fractal hypothesis states:

\begin{center}
\textit{The Steven Elliott Equations, derived from quaternion electromagnetism, \\
exhibit exact or approximate scale invariance.}
\end{center}

Mathematically, under the scaling transformation:
\begin{align}
\mathbf{r} &\to \lambda \mathbf{r} \\
t &\to \lambda t \\
\rho_q &\to \lambda^{-3} \rho_q \\
\mathbf{F} &\to \lambda^{-2} \mathbf{F} \\
\eta &\to \lambda \eta
\end{align}

the equations maintain their form (up to rescaling of coefficients).

\subsection{The Universal Scale Factor $\lambda$}

Empirically, comparing atomic to galactic scales suggests:
\begin{equation}
\lambda \approx 10^{33}
\end{equation}

This is derived from ratios like:
\begin{align}
\frac{R_{\text{galaxy}}}{R_{\text{atom}}} &\sim 10^{31} \\
\left(\frac{M_{\text{galaxy}}}{M_{\text{atom}}}\right)^{1/3} &\sim 10^{34}
\end{align}

Taking a geometric mean gives $\lambda \approx 10^{33}$ as the characteristic scale step between adjacent "levels" of the fractal hierarchy.

\subsection{The Regime Boundary}

At each scale, there is a characteristic length $\sim 0.1$ mm where the dominant forces transition. In standard physics:
\begin{itemize}
    \item Below 0.1 mm: electromagnetic forces dominate
    \item Above 0.1 mm: gravitational forces dominate (at macro scales)
\end{itemize}

In SEEF, this is reinterpreted as a transition in the \textit{Reynolds number regime} of the quaternion fluid:
\begin{equation}
Re_q = \frac{\rho_q v_q L}{\eta_q}
\end{equation}

Below the boundary: laminar, EM-dominated flow \\
Above the boundary: turbulent, pressure-gradient-dominated flow

Scaled by $\lambda$, the 0.1 mm boundary at our scale corresponds to galactic halo sizes at scale $N-1$.

\subsection{Discrete Scale Hierarchy}

The fractal structure consists of discrete levels:
\begin{itemize}
    \item \textbf{$N = 0$}: Our observable universe
    \item \textbf{$N + 1$}: Atomic/molecular scale (our galaxies are atoms here)
    \item \textbf{$N - 1$}: Super-galactic scale (our atoms are galaxies here)
    \item etc., extending infinitely in both directions
\end{itemize}

At each level, the Steven Elliott Equations govern dynamics with parameters scaled by $\lambda$.

\newpage

\section{Part III: Emergent Phenomena from Fractal NS}

\subsection{Gravity as Large-Scale Fluid Pressure}

In the Steven Elliott framework, what we perceive as "gravity" is the pressure-gradient term in the quaternion NS equation operating at scale $N-1$:

\begin{equation}
\mathbf{f}_{\text{grav}} \sim -\nabla p_q
\end{equation}

When $N-1$ scale quaternion fluids (our "atoms" = their "galaxies") interact, the pressure gradients produce an effective attractive force that mimics Newtonian gravity.

\textbf{This is NOT "QED interference"—it is fluid pressure dynamics.}

\subsection{Dark Matter as Turbulent Flow Structure}

At scale $N+1$, our galaxies are atoms. The "dark matter halo" is the turbulent flow structure around the dense nucleus (visible matter).

In NS turbulence:
\begin{itemize}
    \item Dense regions create vortices and eddies
    \item These flow structures extend beyond the visible object
    \item They carry momentum and produce forces on nearby objects
\end{itemize}

Dark matter rotation curves = turbulent velocity profiles in the quaternion fluid at scale $N+1$.

\subsection{Quantum Mechanics as Microscale NS Turbulence}

At scale $N+1$ (atomic scale from our perspective), the chaotic/probabilistic nature of quantum mechanics may emerge from turbulent NS dynamics:

\begin{itemize}
    \item Wavefunction $\psi$ = turbulent velocity potential $\phi$ in the quaternion fluid
    \item Heisenberg uncertainty = fundamental turbulent fluctuations at small scales
    \item Quantum tunneling = pressure-driven flow through potential barriers
    \item Entanglement = correlated turbulent structures
\end{itemize}

\subsection{Vacuum Energy from Substrate Flow}

The "vacuum" at scale $N$ is not empty—it contains the quaternion fluid at scale $N-1$. The vacuum energy density is:

\begin{equation}
\rho_{\Lambda} \sim \frac{\langle \rho_{q}(N-1) \rangle}{\lambda^4}
\end{equation}

This gives the right order of magnitude for the cosmological constant without fine-tuning.

\newpage

\section{Part IV: Computational Validation Strategy}

\subsection{Overview: Simulation Before Observation}

The key advantage of SEEF is that it can be validated through direct numerical simulation rather than waiting for telescope observations or expensive experiments.

The plan:
\begin{enumerate}
    \item Implement a quaternion CFD solver for the Steven Elliott Equations
    \item Run matched simulations at multiple scales
    \item Compare outputs to known astrophysical and quantum data
    \item Refine parameters if needed
    \item Use validated model to make predictions
\end{enumerate}

\subsection{Simulation 1: Galactic Rotation Curves}

\textbf{Setup:}
\begin{itemize}
    \item Initialize a rotating disk of quaternion fluid (scale $N+1$)
    \item Set up boundary conditions matching a typical spiral galaxy
    \item Evolve the system and measure velocity profiles
\end{itemize}

\textbf{Success criterion:}
\begin{itemize}
    \item Velocity curves should flatten at large radii (dark matter effect)
    \item No need to add external "dark matter particles"
    \item Turbulent flow structures should naturally produce observed profiles
\end{itemize}

\textbf{Key parameters to tune:}
\begin{itemize}
    \item Quaternion viscosity $\eta_q$
    \item Reynolds number at galactic scale
    \item Initial rotation rate
\end{itemize}

\subsection{Simulation 2: Gravitational Lensing from Fluid Density}

\textbf{Setup:}
\begin{itemize}
    \item Create two dense quaternion fluid regions (galaxies at $N+1$)
    \item Trace light ray paths through the varying fluid density
    \item Calculate lensing deflection angles
\end{itemize}

\textbf{Success criterion:}
\begin{itemize}
    \item Lensing deflections should match observations (including "dark matter" contribution)
    \item Fluid density gradients should bend light correctly
\end{itemize}

\subsection{Simulation 3: N-Body Analog with Fluid Elements}

\textbf{Setup:}
\begin{itemize}
    \item Replace traditional N-body gravitational simulation with quaternion fluid elements
    \item Initialize with galaxy cluster initial conditions
    \item Evolve via Steven Elliott Equations instead of $F = GMm/r^2$
\end{itemize}

\textbf{Success criterion:}
\begin{itemize}
    \item Cluster dynamics should match observations
    \item Bullet Cluster-like collisions should show fluid separation (visible vs. "dark")
    \item No need for separate dark matter component
\end{itemize}

\subsection{Simulation 4: Quantum Scattering as Micro-Scale NS}

\textbf{Setup:}
\begin{itemize}
    \item Simulate electron-electron scattering at scale $N+1$
    \item Treat electrons as localized quaternion fluid packets
    \item Evolve collision dynamics via Steven Elliott Equations
\end{itemize}

\textbf{Success criterion:}
\begin{itemize}
    \item Scattering cross-sections should match QED predictions
    \item Turbulent interactions should produce quantum-like probabilities
    \item Conservation laws should hold
\end{itemize}

\subsection{Simulation 5: Vacuum Energy / Casimir Force}

\textbf{Setup:}
\begin{itemize}
    \item Place two conducting plates in quaternion vacuum (substrate flow from $N-1$)
    \item Calculate pressure difference from restricted flow modes
\end{itemize}

\textbf{Success criterion:}
\begin{itemize}
    \item Force should scale as $F \sim 1/d^4$ (Casimir)
    \item Magnitude should match experimental values
\end{itemize}

\subsection{Simulation 6: Turbulent Cascade Across Scales}

\textbf{Setup:}
\begin{itemize}
    \item Run multi-scale simulation from atomic to galactic
    \item Inject energy at intermediate scale
    \item Track energy cascade up and down the fractal hierarchy
\end{itemize}

\textbf{Success criterion:}
\begin{itemize}
    \item Energy should cascade according to Kolmogorov-like spectrum
    \item Fractal self-similarity should be evident in structure functions
    \item No artificial scale breaks should appear
\end{itemize}

\subsection{Computational Requirements}

\textbf{Hardware:}
\begin{itemize}
    \item GPU cluster for 3D quaternion CFD
    \item Need O($10^6$ - $10^9$) grid points for turbulent resolution
    \item Adaptive mesh refinement (AMR) to handle multi-scale features
\end{itemize}

\textbf{Software:}
\begin{itemize}
    \item Custom quaternion NS solver (extend existing CFD codes)
    \item Parallelized with MPI/OpenMP
    \item Visualization toolkit for quaternion fields
\end{itemize}

\textbf{Timeline:}
\begin{itemize}
    \item 2026 Q1-Q2: Develop and validate solver on simple test cases
    \item 2026 Q3-Q4: Run Simulations 1-3 (galactic scale)
    \item 2027 Q1-Q2: Run Simulations 4-5 (quantum scale)
    \item 2027 Q3-Q4: Run Simulation 6 (multi-scale cascade)
    \item 2028: Publish results and compare to observations
\end{itemize}

\newpage

\section{Part V: Ontological Framework}

\subsection{What is "Real" in SEEF?}

In classical physics:
\begin{itemize}
    \item \textbf{Real}: Point particles with positions and momenta
    \item \textbf{Derived}: Fields, fluids, emergent phenomena
\end{itemize}

In SEEF:
\begin{itemize}
    \item \textbf{Real}: Quaternion fluid continuum at all scales
    \item \textbf{Derived}: "Particles," "forces," discrete objects
\end{itemize}

The universe is fundamentally a fractal fluid. What we call particles, galaxies, atoms—these are coherent structures (vortices, solitons, turbulent features) within the quaternion NS flow.

\subsection{Space and Time}

\textbf{Space:} Flat 3D Euclidean space at all scales. No spacetime curvature.

\textbf{Time:} Universal parameter $t$ that flows at the same rate across scales (after appropriate rescaling by $\lambda$).

Gravitational time dilation arises from local variations in quaternion fluid density, which affect the "effective speed" of signal propagation (analogous to sound speed varying with density in ordinary fluids).

\subsection{Forces}

There are no fundamental forces. What we perceive as forces are:

\begin{itemize}
    \item \textbf{Electromagnetic}: Direct quaternion fluid interactions at our scale
    \item \textbf{Gravitational}: Pressure gradients in the $N-1$ scale fluid
    \item \textbf{Strong/Weak}: Fluid interactions at $N+1$ scale (to be developed)
\end{itemize}

All forces reduce to terms in the Steven Elliott Equations at the appropriate scale.

\subsection{Constants of Nature}

Constants like $G$, $\hbar$, $c$, $e$ are not fundamental—they are emergent from:
\begin{itemize}
    \item The scale factor $\lambda$
    \item The regime boundary position ($\sim 0.1$ mm)
    \item The quaternion fluid properties ($\eta_q$, $p_q$, etc.)
\end{itemize}

In principle, these "constants" could be calculated from first principles if we knew the detailed structure of the quaternion fluid.

\subsection{Determinism and Probability}

The Steven Elliott Equations are deterministic PDEs. Quantum probability emerges from:
\begin{itemize}
    \item Sensitive dependence on initial conditions (chaos)
    \item Turbulent fluctuations at small scales
    \item Coarse-graining / incomplete information about the full quaternion state
\end{itemize}

This is analogous to how statistical mechanics emerges from deterministic classical mechanics.

\newpage

\section{Part VI: Relation to Existing Theories}

\subsection{General Relativity}

GR describes gravity as spacetime curvature. SEEF replaces this with:
\begin{itemize}
    \item Flat space (no curvature)
    \item Quaternion fluid pressure gradients produce "gravitational" effects
    \item Time dilation from fluid density variations (not metric curvature)
\end{itemize}

\textbf{Advantage:} No singularities (black holes are just high-density fluid regions). No need for exotic dark energy.

\textbf{Challenge:} Must reproduce all GR tests (perihelion precession, gravitational waves, etc.) from fluid dynamics.

\subsection{Quantum Mechanics}

QM is probabilistic and uses wavefunction formalism. SEEF reinterprets:
\begin{itemize}
    \item Wavefunction $\psi$ = quaternion fluid velocity potential
    \item Schrödinger equation = linearized NS equation in certain regime
    \item Measurement = turbulent collapse of fluid structures
\end{itemize}

\textbf{Advantage:} Restores determinism. Explains quantum weirdness as turbulent fluid behavior.

\textbf{Challenge:} Must reproduce all QM predictions (interference, entanglement, etc.) from NS dynamics.

\subsection{Standard Model}

The Standard Model has many particles and coupling constants. SEEF suggests:
\begin{itemize}
    \item Particles = stable vortex structures in quaternion fluid
    \item Coupling constants = ratios of fluid parameters at different scales
    \item Gauge symmetries = invariances of the quaternion NS equations
\end{itemize}

\textbf{Advantage:} Unified framework. Fewer fundamental entities.

\textbf{Challenge:} Detailed mapping from fluid structures to SM particles is incomplete.

\newpage

\section{Conclusion and Next Steps}

\subsection{Summary of SEEF}

\begin{enumerate}
    \item \textbf{Foundation:} Rederive Navier-Stokes from Maxwell quaternions → Steven Elliott Equations
    \item \textbf{Hypothesis:} These equations exhibit fractal self-similarity across scales
    \item \textbf{Implication:} Gravity, dark matter, QM emerge from scale-invariant fluid dynamics
    \item \textbf{Validation:} Run simulations to check if fractal NS reproduces observations
\end{enumerate}

\subsection{Immediate Next Steps (2026)}

\begin{itemize}
    \item \textbf{January-March:} Implement quaternion CFD solver
    \item \textbf{April-June:} Validate on known fluid dynamics problems
    \item \textbf{July-September:} Run first galactic rotation curve simulations
    \item \textbf{October-December:} Compare to observational data; iterate
\end{itemize}

\subsection{Open Questions}

\begin{itemize}
    \item What is the precise form of the quaternion viscosity $\eta_q$?
    \item Can we derive the value of $\lambda \approx 10^{33}$ from first principles?
    \item How do we handle the strong/weak forces in this framework?
    \item What is the microscopic structure of the quaternion fluid?
\end{itemize}

\subsection{Philosophical Implications}

If SEEF is correct:
\begin{itemize}
    \item The universe is fundamentally simpler than current physics suggests
    \item All of physics reduces to one equation (Steven Elliott Equations) at different scales
    \item The appearance of distinct forces and particles is an artifact of scale separation
    \item Turbulence and chaos are not bugs—they are fundamental features of reality
\end{itemize}

\vspace{1cm}

\noindent \textbf{Contact:} \\
\href{mailto:seeyallc6c@gmail.com}{seeyallc6c@gmail.com}

\end{document}
