\documentclass{article}
\usepackage{amsmath, amssymb, amsthm}
\usepackage{thmtools}
\usepackage{geometry}
\geometry{margin=1in}

\newtheorem{theorem}{Theorem}
\newtheorem{lemma}[theorem]{Lemma}
\newtheorem{corollary}[theorem]{Corollary}

\title{On the Non-Existence of Closed-Form Ray Trajectories\\ in the Smooth Tetrahedral 4-Reflective-Sphere Billiard}
\author{}

\begin{document}
\maketitle

\begin{theorem}[No closed-form for generic light rays]
  Let four disjoint closed 2-spheres \( S_1, S_2, S_3, S_4 \) in \( \mathbb{R}^3 \) form a ``tetrahedral'' configuration (mutually externally tangent or nearly tangent, with no triple intersections), and let an external sphere \( W \) enclose them with \( S_i \subset \operatorname{int} W \) for \( i=1,2,3,4 \). Consider the optical billiard where a ray starts on \( W \) with inward direction and reflects specularly on the inner spheres until it either escapes back to \( W \) or continues indefinitely.

  Then, for generic initial data (position on \( W \), inward unit direction), the infinite trajectory admits \emph{no closed-form expression in finite radicals or elementary functions} of the initial conditions.
\end{theorem}

\begin{lemma}[Cofinal Turing-universality of polyhedral approximations]
  For any Turing machine \( \mathcal{M} \) with finite state set and alphabet, there exists \( N_0 \in \mathbb{N} \) such that for all \( N \geq N_0 \), there exists a configuration of five \( N \)-sided convex polyhedra:
  \[
    P_1^{(N)},\, P_2^{(N)},\, P_3^{(N)},\, P_4^{(N)},\, P_W^{(N)},
  \]
  approximating the four inner spheres \( S_i \) and the outer sphere \( W \), with the following properties:
  \begin{enumerate}
    \item All \( P_i^{(N)} \) are strictly convex, pairwise disjoint, and contained in \( \operatorname{int} P_W^{(N)} \).
    \item The billiard dynamics (specular reflection on flat faces) can be encoded into symbolic dynamics such that:
      \begin{itemize}
        \item Each face or face group is assigned a label (state, head position, tape symbol) corresponding to \( \mathcal{M} \).
        \item A sequence of consecutive face hits corresponds to a sequence of TM states and transitions.
      \end{itemize}
    \item \( \mathcal{M} \) halts if and only if the trajectory eventually hits \( P_W^{(N)} \) from the inside (escape).
    \item \( \mathcal{M} \) runs infinitely if and only if the trajectory remains trapped among \( P_1^{(N)}, \dots, P_4^{(N)} \) (infinite orbit).
  \end{enumerate}
  Moreover, as \( N \to \infty \), the polyhedral surfaces can be chosen so that:
  \[
    \partial P_i^{(N)} \to S_i \quad \text{and} \quad \partial P_W^{(N)} \to W
  \]
  in the \( C^k \) norm (for any fixed \( k \)), with uniformly convergent normal fields. Under this convergence, finite-time segments of trajectories converge in the Hausdorff sense on path space.
\end{lemma}

\begin{proof}[Proof of Theorem]
  Argue by contradiction. Suppose that, for generic initial data \( (p,v) \in W \times S^2_{\text{inward}} \), there exists a closed-form expression for the infinite ray trajectory in terms of finite radicals and elementary functions.

  \paragraph{Step 1: Projection and next-intersection equation.}
  Parameterize the initial ray from \( p \in W \) with inward direction \( v \). Apply a fixed Möbius inversion centered at \( p \) (or a nearby point), mapping the ray into a half-line in \( \mathbb{R}^3 \simeq \mathbb{C} \times \mathbb{R} \). The four spheres \( S_1, \dots, S_4 \) and the outer sphere \( W \) are mapped to five smooth surfaces; the next intersection with any of them is determined by solving
  \[
    \gamma(t) = p + t v \in \bigcup_{i=1}^4 S_i \cup W.
  \]
  In the projected complex chart, the next intersection coordinate \( z \) satisfies a resultant equation built from the four quadratic defining equations of the \( S_i \) and the external “halting” condition (escape to \( W \)).

  \paragraph{Step 2: Degree-5 resultant and generic irreducibility.}
  The combination of the four quadratic sphere conditions and the external inversion term, together with the full direction and angle information retained as hidden variables in the iteration, leads generically to a one-variable polynomial equation of degree 5:
  \[
    R(z) = \sum_{k=0}^5 a_k \, z^k = 0, \quad a_k \in \mathbb{C},
  \]
  where the coefficients depend algebraically on the initial data \( (p,v) \). For generic \( (p,v) \), this quintic is irreducible over the radical closure of the initial data, and its Galois group is generically \( S_5 \) or \( A_5 \), which are non-solvable.

  By the Abel–Ruffini theorem, the next reflection point cannot be expressed in finite radicals for generic initial data.

  \paragraph{Step 3: Algebraic obstruction to closed-form trajectories.}
  An infinite trajectory is an infinite sequence of reflection points \( r_1, r_2, r_3, \dots \), each determined by solving a new quintic whose coefficients depend algebraically on the previous reflection. If the full trajectory had a closed-form expression in finite radicals and elementary functions, this would imply that an infinite sequence of branch choices could be uniformly prescribed by a finite radical tower. But the chain of generically irreducible quintics has non-solvable Galois groups, so no such finite closed form exists for generic infinite orbits.

  \paragraph{Step 4: Computational obstruction via Turing-universality.}
  Now suppose, for contradiction, that there exists a \emph{computable} function
  \[
    f \colon \text{initial data} \mapsto \text{infinite symbolic sequence and escape/entrance}
  \]
  given in closed form (elementary functions). By the Lemma, for any TM \( \mathcal{M} \), there exists \( N_0 \) such that for all \( N \geq N_0 \), the corresponding \( N \)-gonal polyhedral billiard encodes \( \mathcal{M} \) so that:
  \[
    \mathcal{M} \text{ halts} \quad \Longleftrightarrow \quad \text{the trajectory eventually escapes to } P_W^{(N)}.
  \]
  Under the uniform \( C^k \) convergence of boundaries and Hausdorff convergence of paths, this correspondence passes to the smooth limit: a sequence that eventually escapes in the polyhedral system corresponds, in the limit, to a sequence that eventually hits \( W \) in the smooth system.

  If \( f \) were computable, one could decide whether \( \mathcal{M} \) halts by evaluating \( f \) and checking whether the symbolic sequence ever hits the “escape” label. But the halting problem is undecidable, so no such computable \( f \) can exist.

  \paragraph{Step 5: Conclusion.}
  The algebraic obstruction (Step 3) shows that no closed-form expression in radicals exists for generic orbits. The computational obstruction (Step 4) strengthens this: even if one attempts to escape the radical-tower limit via infinite series or conditional branches, the resulting expression would still have to be computable, which is ruled out by the reduction from the halting problem. Therefore, generic infinite ray trajectories in this smooth tetrahedral 4-reflective-sphere billiard have no closed-form expression in finite radicals or elementary functions of the initial conditions.
\end{proof}

\section*{Conjectural Implications}

\begin{itemize}
  \item \textsc{The system as a one-way function and \( P \neq NP \):}
    The map \( \Phi \colon (p,v) \mapsto (\text{escape time}, \text{symbolic sequence}) \) is easy to compute forward (numerically) but hard to invert (predicting exact sequence or halt time is undecidable in the limit). This is a natural one-way function candidate. If \( P = NP \), all such problems (including halting reductions) would be solvable in polynomial time, contradicting the undecidability argument. Thus, the existence of such a physical one-way function implies \( P \neq NP \).

  \item \textsc{Informational singularity in Navier--Stokes via sphere packings:}
    Replace light rays with a many-body gas of small spheres in a tetrahedral packing bounded by \( W \). The collision dynamics approximate the optical billiard, and in the permuton limit, converge to incompressible/compressible Navier--Stokes flows. The TM encoding persists in the discrete system and carries over to the PDE limit, so there are initial vorticity/spin configurations where long-term behavior is undecidable (e.g., “does a vortex reach region \( X \)?”) despite classical determinism. This is an \emph{informational singularity}: perfect knowledge of initial data is insufficient to decide the future, breaking naive determinism.

  \item \textsc{NS PDEs and an effective Heisenberg principle:}
    Through the Madelung transform and related analogies, incompressible Navier--Stokes can be mapped to a Schr\"odinger-type system with a quantum potential. The Turing-completeness of NS steady/transient flows implies that long-term prediction is fundamentally limited, not just by finite precision, but by zero-precision undecidability. Thus, in blow-up regimes, the PDE exhibits an effective ``uncertainty'':
    \[
      \Delta x \Delta p \gtrsim \epsilon,
    \]
    where \( \epsilon \) reflects an information-theoretic scale, analogous to the Heisenberg relation \( \Delta x \Delta p \ge \hbar/2 \). This suggests that quantum indeterminacy is encoded in the structure of classical continuum PDEs in highly chaotic regimes.
\end{itemize}

These three implications are simultaneously satisfied by the structure of the billiard and its Navier--Stokes analog: the one-way function from chaos, the informational singularity from Turing-completeness, and the effective uncertainty from the wave-like PDE structure.
\end{document}
