\documentclass[11pt]{article}
\usepackage{amsmath,amssymb,physics,geometry}
\usepackage{hyperref}
\geometry{margin=1in}

\title{The Fractal Substrate Equivalence Principle II: \
Boundary-Dominated Dynamics, Scaling, and Phenomenology}
\author{[Author Name]}
\date{}

\begin{document}
\maketitle

\begin{abstract}
We present a phenomenological extension of the Fractal Substrate Equivalence Principle (FSEP), in which atomic, mesoscopic, and galactic structures arise as boundary-dominated dynamics of a scale-extendable electrical substrate. The substrate is modeled as the continuum limit of charged billiard dynamics whose invariant sets form statistically self-similar (Apollonian-like) sphere packings. Rather than postulating fundamental fields or forces, classical field theories and gravitation emerge as coarse-grained descriptions of dissipative boundary flows. We show that regime-change boundaries generically exhibit noisy constrained-gradient dynamics equivalent to simulated annealing. Representative scaling relations are presented, including a geometric interpretation of the fine-structure constant and composition-dependent effective cutoff lengths. Empirical geophysical and astrophysical observations are summarized as quantitative constraints. All claims are explicitly framed as phenomenological and falsifiable.
\end{abstract}

\section{Scope, Status, and Claims}
This work presents a phenomenological framework extension rather than a completed fundamental theory.

\textbf{Included claims:}
\begin{itemize}
\item Physical structure is dominated by boundary dynamics across scales.
\item Effective forces arise from scale-truncated descriptions of a single substrate.
\item Optimization-like dynamics are unavoidable at regime-change boundaries.
\end{itemize}

\textbf{Excluded claims:} abolition of GR/QFT, global entropy reduction, or agency as a physical observable.

\section{Substrate Ontology: Electrical Billiard Limit}
We postulate an underlying electrical information substrate defined as the infinite-scale limit of charged billiard dynamics. The billiards represent infinitesimal degrees of freedom whose recursive coarse-graining yields meta-elements at all scales. No universal minimum length is assumed.

\section{Geometry: Apollonian Packings}
Apollonian sphere packings exhibit discrete scale invariance, conformal self-similarity, and concentration of measure on interfaces. They possess a non-integer fractal dimension $D$ governing the scaling
\begin{equation}
N(r) \sim r^{-D},
\end{equation}
where $N(r)$ is the number of spheres or voids larger than scale $r$. We propose that this fractal dimension imprints scale-free power laws observed across physical regimes.

\section{Boundary Dynamics and Effective MHD}
Coarse-grained boundary statistics admit an effective magnetohydrodynamic or Navier--Stokes-like description. These equations are effective macroscopic closures, not fundamental laws; dissipation parameters emerge from scale truncation and environmental coupling.

\section{Regime-Change Boundaries and Optimization}
Boundary layers generically combine strong gradients, geometric constraints, chaotic fluctuations, and irreversible dissipation. Under these conditions, boundary evolution approximates noisy constrained-gradient flow:
\begin{equation}
\dot{\mathbf{x}} = -\nabla \mathcal{F}(\mathbf{x}) + \boldsymbol{\eta}(t),
\end{equation}
with stationary distribution
\begin{equation}
P(\mathbf{x}) \propto \exp[-\mathcal{F}(\mathbf{x})/T_{\mathrm{eff}}].
\end{equation}
This is mathematically equivalent to simulated annealing and remains thermodynamically consistent: local entropy reduction accompanies net entropy production.

\section{Computation and Universality}
Chaotic boundary dynamics with unbounded state resolution are conjectured to admit computational universality. This claim concerns capacity only and does not assert semantic meaning or agency.

\section{Emergent Forces}
Effective forces arise from parameterizations of boundary dynamics. Electromagnetism corresponds to local circulation and charge-density gradients, while gravitation corresponds to long-wavelength pressure imbalance and drag. These are scale-dependent descriptions of the same substrate.

\section{Quantitative Phenomenology}
\subsection{Fine-Structure Constant (Heuristic)}
At electrical boundary layers, competing dimensional contributions arise:
\begin{itemize}
\item volumetric circulation ($\sim \pi^3$),
\item planar stabilization at interfaces ($\sim \pi^2$),
\item line-like transport along fractal boundaries ($\sim \pi$).
\end{itemize}
In an Apollonian packing, four generating spheres define the minimal volumetric unit, yielding a volumetric contribution $4\pi^3$. Summing dimensional contributions gives the heuristic relation
\begin{equation}
4\pi^3 + \pi^2 + \pi \approx 137.036,
\end{equation}
coincident with the observed inverse fine-structure constant $\alpha^{-1} \approx 137.036$. This relation is not a first-principles derivation but a geometric consistency condition suggesting that $\alpha^{-1}$ encodes the relative weighting of volumetric, surface, and line contributions at boundary layers.

\subsection{Composition-Dependent Effective Scales}
In the absence of a universal minimum length, effective cutoff scales depend on local composition and packing density. Regions dominated by heavy nuclei correspond to denser boundary packings and smaller effective cutoffs, reproducing environment-sensitive coupling behavior.

\subsection{Planetary Cores}
Dense planetary cores are interpreted as large-scale regime-change boundaries dominated by heavy-element packings. The framework permits weak, steady transmutation and enhanced dissipation without violation of conservation laws.

\subsection{Empirical Constraints}
Relevant observations include terrestrial heat flow ($\sim47$ TW), geoneutrino bounds from KamLAND and Borexino, and persistent low-level neutron backgrounds. These observations serve as upper bounds on allowable boundary-driven processes.

\section{Predictions and Falsifiability}
The framework is falsifiable via:
\begin{enumerate}
\item environment-dependent deviations in effective coupling constants,
\item correlations between dissipation rates and boundary geometry,
\item observational bounds on anomalous heat, neutrino, or transmutation signals.
\end{enumerate}

\section{Discussion and Limitations}
The billiard-to-boundary limit remains conjectural, and the dynamics are phenomenological rather than derived from a variational principle. Biological and cognitive interpretations are outside the scope of this work.

\section{Conclusion}
FSEP II reformulates physical unification as a boundary-dominated, scale-extendable, dissipative framework consistent with non-equilibrium thermodynamics. By separating ontology, dynamics, and interpretation, the theory becomes accessible to formal critique and experimental constraint.

\end{document}
